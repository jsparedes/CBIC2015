\documentclass[tikz]{standalone}
\usepackage[latin9]{inputenc}
\usetikzlibrary{automata,positioning}

\begin{document}
	\begin{tikzpicture}
	\filldraw[fill=red!40!orange, draw=black] (0,0) rectangle (5,5) node[pos=.5, text=white, font=\bfseries] {\huge Agrega��o};
	\filldraw[fill=black!90!white, draw=black] (0,5) rectangle (5,10) node[pos=.5, text width=3cm, text=white]{\Large Uni�o das ativa��es de cada regra};
	\end{tikzpicture}
\end{document}


%\documentclass[tikz]{standalone}
%\usepackage[latin9]{inputenc}
%\usetikzlibrary{automata,positioning}
%
%\usetikzlibrary{shapes,arrows}
%
%\begin{document}
%\tikzset{block1/.style={rectangle, draw, fill=black, text width=8em, text centered, %minimum height=8em, minimum width=8em}}
%\tikzset{block2/.style={rectangle, draw, fill=red!40!orange, text width=5em, text %centered, minimum height=8em, minimum width=8em}}
%\tikzset{whttxt/.style={text=white, font=\bfseries}}
%
%\begin{tikzpicture}[node distance=1in]
%\node [block1, whttxt] (nodedescription) {\scriptsize Aunamiento de premisas para %una clase, generaci�n de la base de reglas para la clase};
%\node [block2, below of=nodedescription, whttxt] (nodestage) {Agrega��o};
%\end{tikzpicture}
%\end{document}